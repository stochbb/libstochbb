\section{Process \& simulation description in XML} \label{sec:xml}
In contrast to the application programming interface (see Section \ref{sec:api}, above), 
complex processes can also be defined in XML and analyzed using the
StochBB command line tool (see Section \ref{sec:cli}, below). A single XML file defines a 
\class{Simulation}, a simple collection of random variables together with the specification which
variables as used for the output. For example the XML code
\begin{lstlisting}[language=XML]
 <?xml version="1.0"?>
 <simulation xmlns="http://hmatuschek.github.io/stochbb/simulation-0.0.dtd"
             xmlns:m="http://www.w3.org/1998/Math/MathML">

  <var type="gamma" id="gamma">
   <param name="k"> <m:cn>5</m:cn> </param>
   <param name="theta"> <m:cn>30</m:cn> </param>
  </var>

  <output from="0" to="100">
   <var ref="gamma"/>
  </output>

 </simulation>
\end{lstlisting}
specifies a trivial simulation.

Here a single gamma-distributed random variable is defined.
It gets the identifier \code{gamma} assigned. With this identifier, it is possible to refer to the random
variable later. The \code{type} attribute specifies the type of the random variable. Here the
build-in type \code{gamma} is used, specifying a gamma-distributed random variable. The shape and
scale parameters, \code{k} and \code{theta}, are specified using the \code{param} elements. 
The \code{name} attribute of each \code{param} element specified the name of the parameter while 
its value is given as an MathML child element of the \code{param} element.

Finally, the \code{output} element specifies which random variables are used for output. Each child
element of the \code{output} element specifies one variable. Either by referencing a previously defined
random variable or by specifying the random variable directly. For example,
\begin{lstlisting}[language=XML]
 <?xml version="1.0"?>
 <simulation xmlns="http://hmatuschek.github.io/stochbb/simulation-0.0.dtd"
             xmlns:m="http://www.w3.org/1998/Math/MathML">

  <output from="0" to="100">
   <var type="gamma">
    <param name="k"> <m:cn>5</m:cn> </param>
    <param name="theta"> <m:cn>30</m:cn> </param>
   </var>
  </output>

 </simulation>
\end{lstlisting}
is fully equivalent to the previous example.

\subsection{Build-in random variables}
There are only very few build-in random variable types. They can be divided into two groups:
\emph{atomic} and \emph{derived} random variables.

\subsubsection{Atomic random variables}
Atomic random variables are random variables which follow a specific
distribution and are independent from any other random variable defined.

\begin{tabular}{l|l|l}
 Type & Parameters & Process description \\ \hline
 \code{delta} & \code{delay} & A constant delay or a process with a fixed waiting time. \\
 \code{uniform} & \code{a}, \code{b} & A process with a uniform-distributed waiting time. \\
 \code{normal} & \code{mu}, \code{sigma} & A process with a normal-distributed waiting. \\
 \code{gamma} & \code{k}, \code{theta} & A process with a gamma-distributed waiting time.
\end{tabular}

Beside these atomic random variables, there are derived random variables which depend on other
random variables. Hence they usually do not take simple parameters as arguments but other 
random variables.

\subsubsection{Chains of random variables}
One of the most important derived random variable is a chain. A chain can be defined as
\begin{lstlisting}[language=XML]
 <var type="chain">
  <var ref="X1"/> <var ref="X2"/> <var ref="X2"/>
 </var>
\end{lstlisting}

Here the random variable \code{X1}, \code{X2} and \code{X3} are chained. Mathematically,
the waiting time of the chain is simply the sum of the waiting times of the random variables
\code{X1}--\code{X3}.

\subsubsection{Maximum of random variables}
Another derived random variable is the \code{maximum} type. This type can be interpreted as a
processes that waits until all referred parallel processes are completed hence the waiting time
of
\begin{lstlisting}[language=XML]
 <var type="maximum">
  <var ref="X1"/> <var ref="X2"/> <var ref="X2"/>
 </var>
\end{lstlisting}
is just the maximum of the random variables \code{X1}--\code{X3}.

\subsubsection{Minimum of random variables}
Likewise, the \code{minimum} derived random variable is simply the minimum of the referred random
variables and represents the waiting time of the fastest parallel processes \code{X1}--\code{X3}:
\begin{lstlisting}[language=XML]
 <var type="minimum">
  <var ref="X1"/> <var ref="X2"/> <var ref="X2"/>
 </var>
\end{lstlisting}

\subsubsection{Mixture of random variables}
The mixture of random variables is defined using the \code{mixture} variable type with a list of
weight-variable pairs. For example
\begin{lstlisting}[language=XML]
 <var type="mixture">
  <weight> <m:cn>1</m:cn> <var ref="X1"/> </weight>
  <weight> <m:cn>2</m:cn> <var ref="X2"/> </weight>
  <weight> <m:cn>1</m:cn> <var ref="X3"/> </weight>
 </var>
\end{lstlisting}

\subsubsection{Compound random variables} 
A compound random variable $Y$ has a distribution $f_{Y|A}(y|A)$ where the parameter $A$ is a random
variable too, with its own distribution $A~g()$. In XML a compound is specified using the 
\code{compound} type by specifying the base type of the distribution using the \code{base} attribute.
Instead of simple parameters, the parameters of an compound are specified as random variables. 
For example, the code
\begin{lstlisting}[language=XML]
 <var id="mu" ...>
  [ ... ]
 </var>
 <var id="sigma" ...>
  [ ... ]
 </var>
 
 <var type="compound" base="normal">
  <param name="mu"> <var ref="mu"/> </param>
  <param name="sigma"> <var ref="sigma"/> </param>
 </var>
\end{lstlisting}
defines a compound-normal distributed random variable where its mean is distributed like the random 
variable \code{mu} and its standard deviation like \code{sigma}. Of course these parameter random 
variables have to be defined earlier.

\subsection{User defined random-variable types}
The limited number of build-in random-variable types would turn the definition of complex stationary random
processes difficult. Hence it is possible to define new types derived from the build-in ones. For example,
the code 
\begin{lstlisting}[language=XML]
 <define id="exp">
  <param id="lambda"/>

  <var type="gamma" id="exp">
   <param name="k"><m:cn>1</m:cn></param>
   <param name="theta">
    <m:apply>
     <m:divide/>
     <m:cn>1</m:cn>
     <m:ci>lambda</m:ci>
    </m:apply>
   </param>
  </var>
 </define>
\end{lstlisting}
defines a new type \code{exp} as a special case of the Gamma distribution as
$\text{Exp}(\lambda) = \Gamma(1,\lambda^{-1})$. The \code{define} tag takes an \code{id} 
attribute specifying the new type identifier. Then, one ore more \code{param} elements are
used to specify the parameters of the new type. 
Here only one parameter named \code{lambda} is defined. Finally a random variable with the same
identifier as the type name is defined which implements the new random variable type.

A more complex example might be
\begin{lstlisting}[language=XML]
 <define id="exgauss">
  <param id="lambda">
  <param id="mu"/>
  <param id="sigma"/>

  <var type="gamma" id="ex">
   <param name="k"><m:cn>1</m:cn></param>
   <param name="theta">
    <m:apply>
     <m:divide/>
     <m:cn>1</m:cn>
     <m:ci>lambda</m:ci>
    </m:apply>
   </param>
  </var>

  <var type="normal" id="gauss">
   <param name="mu"> <m:ci>mu</m:ci> </param>
   <param name="sigma"> <m:ci>sigma</m:ci> </param>
  </var>

  <var type="chain" id="exgauss">
    <var ref="ex"/> <var ref="gauss"/>
  </var>
 </define>
\end{lstlisting}

This defines a new random-variable type with an exponentially modified Gaussian (Ex-Gauss) 
distribution by chaining an exponential- (defined here as a special case of the Gamma 
distribution) and a Gaussian-distributed random variable. This new type can then be used
to instantiate Ex-Gauss-distributed random variables as
\begin{lstlisting}[language=XML]
 <var type="exgauss">
  <param name="lambda"> <m:cn>10</m:cn> </param>
  <param name="mu"> <m:cn>20</m:cn> </param>
  <param name="sigma"> <m:cn>10</m:cn> </param>
 </var>
\end{lstlisting}

\subsection{Modules}
Some of the aforementioned user defined random variable types may be reused in several
simulations. For these cases it is convenient to export these definitions into separate files
and load them when needed. These files are called modules and form simple collections of
variable type definitions. For example a module defining the an exponential distributed 
random variable could be implemented as
\begin{lstlisting}[language=XML]
 <?xml version="1.0"?>
 <module xmlns="http://hmatuschek.github.io/stochbb/module-0.0.dtd"
             xmlns:m="http://www.w3.org/1998/Math/MathML">

  <define id="exp">
   <param id="lambda"/>

   <var type="gamma" id="exp">
    <param name="k"><m:cn>1</m:cn></param>
    <param name="theta">
     <m:apply>
      <m:divide/>
      <m:cn>1</m:cn>
      <m:ci>lambda</m:ci>
     </m:apply>
    </param>
   </var>
  </define>

 </module>
\end{lstlisting}

Within the body of the module, the user-defined types can be specified. Such a module can then
be loaded using the \code{load} tag within a simulation or other modules like
\begin{lstlisting}[language=XML]
 <?xml version="1.0"?>
 <simulation xmlns="http://hmatuschek.github.io/stochbb/simulation-0.0.dtd"
             xmlns:m="http://www.w3.org/1998/Math/MathML">

  <load>exp.xml</load>

  <output ...>
   [ ... ]
  </output>
 </simulation>
\end{lstlisting}

With this XML language, it is possible to specify complex systems of random variables that can be analyzed 
using the \code{stochbb} command line tool described in the next section.
