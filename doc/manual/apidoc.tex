\section{Brief API Documentation} \label{sec::apidoc}
Beside the brief API discussion above (see Section \ref{sec:api}) this appendix
provides some details on the available classes and  functions defined as the
application programming interface of StochBB. You can find a complete API
reference including the documentation of internal used object at \citep{stochbbapi}.

There is also a Python package called \code{stochbb} providing access to the C++
API in Python. The Python API is almost identical to the C++ one, hence this 
API documentation can be used for both. Differences between the APIs (Python 
does not provide constructs like templates) are documented.

\begin{deffunc}{stochbb::delta}{\class{Var}}{double value}	
 Constructs a new delta-distributed random variable located at \code{value}, that is $\delta(x-x_0)$.
\end{deffunc}

\begin{deffunc}{stochbb::uniform}{\class{Var}}{double a, double b}	
Constructs a new uniform-distributed random variable on the interval $[a,b]$, that is $\text{U}[a,b](x)$.
\end{deffunc}

\begin{deffunc}{stochbb::normal}{\class{Var}}{double mu, double sigma}
Constructs a new normal-distributed random variable with mean \code{mu} and 
standard deviation \code{sigma} , that is $\phi(x;\mu, \sigma)$.
\end{deffunc}

\begin{deffunc*}{stochbb::normal}{\class{Var}}{const \class{Var}\& mu, const \class{Var}\& sigma}
Constructs a new compound-normal-distributed random variable with mean \code{mu} and 
standard deviation \code{sigma} being random variables too.
\end{deffunc*}

\begin{deffunc}{stochbb::gamma}{\class{Var}}{double k, double theta}
Constructs a new gamma-distributed random variable with shape \code{k} and 
scale \code{theta} , that is $\Gamma(x; k, \theta)$.
\end{deffunc}

\begin{deffunc*}{stochbb::gamma}{\class{Var}}{const \class{Var}\& k, const \class{Var}\& theta}
Constructs a new compound-gamma-distributed random variable with shape \code{k} and 
scale \code{theta} begin random variables too.
\end{deffunc*}

\begin{deffunc}{stochbb::weibull}{\class{Var}}{double k, double lambda}
Constructs a new Weibull-distributed random variable with shape \code{k} and 
scale \code{lambda} , that is $Weibull(x; k, \lambda)$.
\end{deffunc}

\begin{deffunc}{stochbb::affine}{\class{Var}}{const \class{Var}\& X, double a, double b}
Constructs an affine transformed random variable from the given one as $Y = a\,X+b$.
\end{deffunc}

\begin{deffunc}{stochbb::chain}{\class{Var}}{const std::vector<\class{Var}>\& variables}
Constructs a chain of several random variables as $Y = \sum_i X_i$. 
\end{deffunc}

\begin{deffunc}{stochbb::maximum}{\class{Var}}{const std::vector<\class{Var}>\& variables}
Constructs a maximum random variable, a variable that represents the maximum of several other random variables as $Y = \max\{X_1,...,X_N\}$.
\end{deffunc}

\begin{deffunc}{stochbb::minimum}{\class{Var}}{const std::vector<\class{Var}>\& variables}
Constructs a minimum random variable, a variable that represents the minimum of several other random variables as $Y = \min\{X_1,...,X_N\}$.
\end{deffunc}

\begin{deffunc}{stochbb::mixture}{\class{Var}}{double w1, const \class{Var}\& X1, double w2, const \class{Var}\& X2}
Constructs a mixture of the random variables $X_1$ and $X_2$ using their associated weights $w_1$ and $w_2$.
\end{deffunc}

\begin{deffunc}{stochbb::mixture}{\class{Var}}{double w1, const \class{Var}\& X1, double w2, const \class{Var}\& X2, double w3, const \class{Var}\& X3}
Constructs a mixture of the random variables $X_1, X_2$ and $X_3$ using their associated weights $w_1, w_2$ and $w_3$.
\end{deffunc}

\begin{deffunc}{stochbb::mixture}{\class{Var}}{const std::vector<double>\& weights, const std::vector<\class{Var}>\& variables}
Constructs a mixture of the given random variables using their associated weights.
\end{deffunc}

\begin{defclass}{Container}
Base class of all container object holding references to managed objects. This class is an essential part of the internal memory 
management system. All classes of the API are derived from this class.

\begin{classsyn}{}
Constructs an empty container.
\end{classsyn}

\begin{defmeth}{Container}{isNull}{bool}{}
Returns \code{true} if the container is empty. That is, if the container does not refer to an actual object.
\end{defmeth}

\begin{defmeth}{Container}{is<Type>}{bool}{}
Returns \code{true} if the container holds a reference to an object of type \code{Type} and \code{false} otherwise. 
This template method can be used to test a container before casting it with \method{Container}{as<Type>}.

The Python API differs for this function as Python does not support constructs like templates. Hence for test if a container
refers to a specific instance type is implemented using normal method of the form \code{isTYPE()} where \code{TYPE} is the 
type name. For example, to test if a container refers to a variable instance, the method \code{isVar()} can be used.
\end{defmeth}

\begin{defmeth}{Container}{as<Type>}{Type}{}
Casts the container to the type \code{Type}. Returns an empty container if cast fails, e.g. if the reference held by the 
container cannot be casted to the given type or is empty.

The Python API differs for this function as Python does not support constructs like templates. Hence for type casts 
methods are provided of the form \code{asTYPE()} where \code{TYPE} is the type name. For example, to cast a container to the type \class{Var}, the method \code{asVar()} can be used.
\end{defmeth}
\end{defclass}


\begin{defclassex}{Var}{Container}
Base class of all random variables. All random variables have a probability density function assigned which can be accessed
using \method{Var}{density} method.

\begin{classsyn}{Var}{} 
Constructs an empty variable container.
\end{classsyn}

\begin{defmeth}{Var}{density}{\class{Density}}{}
Returns the density of the random variable.
\end{defmeth}

\begin{defmeth}{Var}{dependsOn}{bool}{const Var\& var}
Returns \code{true} if the random variable depends on the given variable \code{var}.
\end{defmeth}

\begin{defmeth}{Var}{mutuallyIndep}{bool}{const Var\& var}
Returns \code{true} if the random variable and the given variable are mutually independent.
\end{defmeth}
\end{defclassex}

\begin{defclassex}{Density}{Container}
Base class of all probability density functions (PDFs).

\begin{defmeth}{Density}{eval}{void}{double min, double max, Eigen::VectorXd\& out}
Evaluates the PDF on a regular grid in $[min, max)$ using the same number of grid points as the number
of elements in \code{out}.
\end{defmeth}
 
\begin{defmeth}{Density}{evalCDF}{void}{double min, double max, Eigen::VectorXd\& out}
Evaluates the cumulative probability function (CDF) on a regular grid in $[min, max)$ using the same number
of grid points as the number of elements in \code{out}.
\end{defmeth}
\end{defclassex}

\begin{defclassex}{DerivedVar}{Var}
Base class of all derived random variables, that are random variables which are defined as functions of other random 
variables.

\begin{defmeth}{DerivedVar}{numVariables}{size\_t}{}
Returns the number of variables the derived variable depends on.
\end{defmeth}

\begin{defmeth}{DerivedVar}{numVariables}{\class{Var}}{size\_t i}
Returns the \code{i}-th random variable the derived variable depends on.
\end{defmeth}
\end{defclassex}

\begin{defclassex}{AffineTafo}{DerivedVar}
This class represents an affine transformed random variable, that is $Y = a\,X+b$. This class does not provide a
public constructor, use the \function{stochbb::affine} function to construct an affine transformation.

\begin{defmeth}{AffineTrafo}{scale}{double}{}
 Returns the scale of the transform, a.k.a. $a$.
\end{defmeth}

\begin{defmeth}{AffineTrafo}{shift}{double}{}
 Returns the shift of the transform, a.k.a. $b$.
\end{defmeth}
\end{defclassex}

\begin{defclassex}{Chain}{DerivedVar}
Implements the sum of the independent random variables $X_i, i=1,\dots,N$, $Y = \sum_i X_i$.

\begin{classsyn}{Chain}{std::vector<\class{Var}>\& variables}
Constructs a random variable being the sum of the given random variables.
Although providing a constructor, a chain should be constructed using the
\function{stochbb::chain} function.
\end{classsyn}
\end{defclassex}

\begin{defclassex}{Maximum}{DerivedVar}
Implements the maximum of the independent random variables $X_i, i=1,\dots,N$, 
$Y = \max\{X_1,...,X_N\}$.

\begin{classsyn}{Maximum}{std::vector<\class{Var}>\& variables}
Constructs a random variable being the maximum of the given random variables.
Although providing a constructor, a maximum should be constructed using the
\function{stochbb::maximum} function.
\end{classsyn}
\end{defclassex}

\begin{defclassex}{Minimum}{DerivedVar}
Implements the minimum of the independent random variables $X_i, i=1,\dots,N$, 
$Y = \min\{X_1,...,X_N\}$.

\begin{classsyn}{Minimum}{std::vector<\class{Var}>\& variables}
Constructs a random variable being the minimum of the given random variables.
Although providing a constructor, a minimum should be constructed using the 
\function{stochbb::minimum} function.
\end{classsyn}
\end{defclassex}

\begin{defclassex}{Mixture}{DerivedVar}
Implements a weighted mixture of several random variables. A mixture can be
seen as a random process that selects one of its children with a certain probability.

\begin{classsyn}{Mixture}{const std::vector<double>\& weights, const std::vector<\class{Var}>\& variables}
Constructs a new mixture random variable from the given vectors of weights and variables. 
\end{classsyn}

\begin{defmeth}{Mixture}{weight}{double}{size\_t i}
Returns the weight of the i-th variable.
\end{defmeth}
\end{defclassex}


\begin{defclassex}{Compound}{DerivedVar}
This class represent the base of all compound random variables, a variable which
is distributed according to a parametric distribution where at least one parameter
is a random variable too, that is $Y\sim f_{X|A}(x|A)$ and $A\sim g(a)$. This
class does not provide a public constructor, please use the constructor functions
like \function{stochbb::norm} and \function{stochbb::gamma} to construct 
compound distributions.
\end{defclassex}


\begin{defclassex}{ExactSampler}{Container}
The \code{ExactSampler} class allows to sample simultaneously from several
possibly mutually dependent random variables. For large systems of dependent
random variables, this sampler can be slow. If only samples from the marginal
distributions are needed, consider using the \class{MarginalSampler} class.

\begin{classsyn}{ExactSampler}{const \class{Var}\& X}
Constructs a sampler for the given variable.
\end{classsyn}

\begin{classsyn}{ExactSampler}{const \class{Var}\& X1, const \class{Var}\& X2}
Constructs a sampler for the given variables.
\end{classsyn}

\begin{classsyn}{ExactSampler}{const \class{Var}\& X1, const \class{Var}\& X2, const \class{Var}\& X3}
Constructs a sampler for the given variables.
\end{classsyn}

\begin{classsyn}{ExactSampler}{const std::vector<\class{Var}>\& variables}
Constructs a sampler for the given variables.
\end{classsyn}

\begin{defmeth}{ExactSampler}{sample}{void}{Eigen::MatrixXd\& out}
Samples from the variables passed to the constructor. Each column in \code{out}
corresponds to a selected variable. The number of samples being drawn is
specified by the number of rows of \code{out}.
\end{defmeth}
\end{defclassex}

\begin{defclassex}{MarginalSampler}{Container}
This class implements an approximate marginal sampler. The sampler first
obtains the CDF of the selected random variable and uses the inverse of a
piece-wise linear interpolation to obtain samples for the random variable. For very
large systems of random variables, this approach may outperform the exact
sampling as implemented by the \class{ExactSampler} class if only samples 
of a single random variable or only samples from marginal distributions are required.

\begin{classsyn}{MarginalSampler}{const Var\& X, double Xmin, double Xmax, size\_t steps}
Constructs a marginal sampler for the random variable $X$ using the CDF evaluated on the interval $[X_{min},X_{max})$ 
with \code{steps} grid points.
\end{classsyn}

\begin{defmeth}{MarginalSampler}{sample}{void}{Eigen::VectorXd\& out}
Samples from the random variable passed to the constructor. The samples are stored in the \code{out} vector where
the number of samples is specified by the number of elements of the vector.
\end{defmeth}
\end{defclassex}
