\section{Application programming interface} \label{sec:api}
The application programming interface (API, see also \cite{stochbbapi}) 
allows to assemble complex systems of random variables programmatically in C++.
All API classes are derived from the \class{Container} class which is an essential part of the
memory management system used by StochBB. Usually the C++ programmer needs to keep track of all
objects still in use and is responsible to free unneeded objects to avoid memory leaks. This can
be a difficult task when dealing with complex structured objects cross-referencing each other.
To ease the usage of StochBB, a \emph{mark and sweep} garbage collector is implemented which keeps track
of all objects being directly or indirectly reachable and freeing all unreachable objects. For
this memory management system to work, it is necessary to treat all container objects like values
although they represent references to objects allocated on the heap.

There is also a Python package (called \code{stochbb}) that exposes the classes and functions 
of the C++ API to python. This allows for a convenient construction of systems of random variables
while maintaining the speed of the C++ classes. The Python API is almost identical to the C++ one
described below except that \code{numpy} array are used in Python instead of Eigen matrices and 
vectors.

The central class of StochBB is \class{Var}, representing a random variable. This could be a
simple random variable having a specified distribution (see \class{AtomicVar}) or a random
variable that is derived from others like \class{Chain}, \class{Minimum}, \class{Maximum} or
\class{Mixture}. All random variables (atomic and derived) have probability density functions
attached. They can be accessed using the \method{Var}{density} method which returns a 
\class{Density} object.

All \class{Density} objects have two methods, \method{Density}{eval} evaluating the probability
density function and \method{Density}{evalCDF} evaluating the cumulative density or probability
function. Assembling a system of random variables and evaluate their PDFs or CDFs is straight
forward. Sampling, however, is not that trivial and is described below in some detail.

\subsection{Assembling a system of random variables}
In a first step, one may define a new gamma-distributed random variable with shape $k=10$
and scale $\theta=100$ as
\begin{lstlisting}[language=C++]
 #include <stochbb/api.h>
 using namespace stochbb;
 
 // [...]
 
 Var X1 = stochbb::gamma(10, 100);
\end{lstlisting}

Its PDF can then be evaluated as on a regular grid in $[0,1000)$ with $1000$ grid points as
\begin{lstlisting}[language=C++]
 // [...]
 
 Eigen::VectorXd pdf(1000);
 X1.density().eval(0, 1000, pdf);
\end{lstlisting}

The result of the evaluation is stored into the vector \code{pdf}. There are only very few basic
or atomic random-variable types defined in StochBB:

\begin{tabular}{l|l|l}
 Constructor & Parameters & Process description \\ \hline
 \function{stochbb::delta} & \code{delay} & A constant delay or a process with a fixed waiting time. \\
 \function{stochbb::unif} & \code{a}, \code{b} & A process with a uniform-distributed waiting time. \\
 \function{stochbb::norm} & \code{mu}, \code{sigma} & A process with a normal-distributed waiting time. \\
 \function{stochbb::gamma} & \code{k}, \code{theta} & A process with a gamma-distributed waiting time. \\
 \function{stochbb::weibull} & \code{k}, \code{lambda} & A process with a Weibull-distributed waiting time. \\
\end{tabular}

More complex processes can be derived by combining these atomic random variables or as special cases of them.


\subsubsection{Affine transformation of random variables}
An affine transformation of the random variable $X$ is of the form $a\,X+b$, where $a\neq 0$ and $b$ are
real values. An affine transformation can be obtained using the overloaded * and + operators or using the
\function{stochbb::affine} function. For example, the code
\begin{lstlisting}[language=C++]
 #include <stochbb/api.h>
 using namespace stochbb;
 
 // [...]
 
 Var X = stochbb::gamma(10, 100);
 Var Y = 3*X + 1;
\end{lstlisting}
constructs a random variable \code{Y} being an affine transformed of the random variable \code{X}.

\subsubsection{Sums of random variables}
Beside the affine transfrom of random variables, the most basic derived random variable is a \class{Chain}.
This type represents the \emph{chaining} of random processes. Such a chain can be constructed using 
the overloaded \code{+} operator or the \function{stochbb::chain} function. For example
\begin{lstlisting}[language=C++]
 #include <stochbb/api.hh>
 using namespace stochbb;

 // [...]

 Var X1 = stochbb::gamma(10,100);
 Var X2 = stochbb::gamma(20, 50);
 Var Y = X1 + X2;
\end{lstlisting}

\subsubsection{Minimum and Maximum of random variables}
Another simple derived random variable is the \class{Maximum} or \class{Minimum} class. As the
names suggest, they represent the maximum or minimum of a set of random variables. They can be
created using the overloaded standard library function \code{std::min} and \code{std::max} or the
\function{stochbb::minimum} and \function{stochbb::maximum} functions. The latter take a vector of random variables.
\begin{lstlisting}[language=C++]
 #include <stochbb/api.hh>
 using namespace stochbb;

 // [...]

 Var X1 = stochbb::gamma(10,100);
 Var X2 = stochbb::gamma(20, 50);
 Var Y = std::max(X1, X2);
\end{lstlisting}

\subsubsection{Mixtures of random variables}
Similar to the \class{Minimum} or \class{Maximum}, a mixture of random variables can be
constructed using the \function{stochbb::mixture} function. This function takes a at least 
two variables and their associated weight. Such a mixture can be considered as a random process which
randomly selects the outcome of a set of other random processes, where the probability of
selecting a specific process is given by the relative weight assigned to each process.
\begin{lstlisting}[language=C++]
 #include <stochbb/api.hh>
 using namespace stochbb;

 // [...]

 Var X1 = stochbb::gamma(10,100);
 Var X2 = stochbb::gamma(20, 50);
 Var Y = stochbb::mixture(1,X1, 2,X2);
\end{lstlisting}

In the example above, the random process $Y$ will select the outcome of $X_1$ with
a probability of $\frac{1}{3}$ and the outcome of $X_2$ with probability
$\frac{2}{3}$.

\subsubsection{Compound random variables}
An important class of derived random processes are compound processes. There the parameters of the
distribution of a random variable are themselves random variables. That is
\begin{equation}
 X \sim f(x|A)\,,\quad A \sim g(a|\theta)\,, \nonumber
\end{equation}
where the random variable $X$ is distributed as $f(x|A)$, parametrized by $A$,
where $A$ itself is a random variable distributed as $g(a|\theta)$, parametrized by
$\theta$. 

Compound random variables are created using the same factory function like the atomic random variable
types. In contrast to the atomic random variables, the factory functions take random variables as 
parameters instead of constant values.
\begin{lstlisting}[language=C++]
 #include <stochbb/api.hh>
 using namespace stochbb;
 
 // [...]
 
 Var mu = stochbb::gamma(10,100);
 Var cnorm = stochbb::norm(mu, 10);
\end{lstlisting}
instantiates a compound-normal distributed random variable, where the mean is gamma distributed
while the standard deviation is fixed.

\subsection{Sampling several dependent random variables}
As mentioned above, sampling efficiently from a complex system of random variables is not trivial. 
First of all, it must be ensured that all atomic random variables are samples only once. Otherwise, two dependent
random variables may be sampled as independent. Moreover, the results of derived random variables
should be cached for efficiency. 

StochBB provides a separate class that implements a proper sampler for a system
of random variables, the \class{ExactSampler} class. This class allows to sample
from several possibly dependent random variables simultaneously. Upon construction,
the set of random variables to sample from, is specified. A sample from these random
variables can then be obtained by the \method{ExactSampler}{sample} method.
\begin{lstlisting}[language=C++]
 #include <stochbb/api.hh>
 using namespace stochbb;

 // [...]

 Var X1 = stochbb::gamma(10,100);
 Var X2 = stochbb::gamma(20, 50);
 Var Y = std::min(X1, X2);
 // Construct sampler
 ExactSampler sampler(X1, X2, Y);
 // Get 1000 samples
 Eigen::MatrixXd samples(3, 1000);
 sampler.sample(samples);
\end{lstlisting}

The \method{ExactSampler}{sample} method takes a reference to a \class{Eigen::Matrix} where each column
represents the random variable given to the constructor and each row an independent sample from
the system.

For very large systems, sampling may get slow. Particularly if one is only interested
in the marginal distribution of single random variables. For these cases, an approximate sampler
for single random variables is provided, the \class{MarginalSampler}. This sampler uses an
approximation of the inverse of the cumulative distribution function of a random variable
to draw samples.
\begin{lstlisting}[language=C++]
 #include <stochbb/api.hh>
 using namespace stochbb;

 // [...]

 Var X1 = stochbb::gamma(10,100);
 Var X2 = stochbb::gamma(20, 50);
 Var Y = std::min(X1, X2);
 // Sample from Y on [0,500] in 1000 steps
 MarginalSampler sampler(Y, 0, 500, 1000);
 // Get 1000 samples
 Eigen::VectorXd samples(1000);
 sampler.sample(samples);
\end{lstlisting}

In this example, a \class{MarginalSampler} is constructed for the random variable \code{Y}. Using an
approximation of its CDF on the interval $[0,500)$ using $1000$ steps. Then, the
\method{MarginalSampler}{sample} method is used to obtain $1000$ independent samples.


